\chapter{Validação experimental, Simulação e Testes}
\label{chap:validacao}

\section{Introdução}
\label{validacao:sec:introducao}

\section{Interface de Visualização}
\label{validacao:sec:interface}
Para demonstrar o funcionamento e o estado do sistema, foi desenvolvida uma interface gráfica. Nesta secção serão descritos os componentes apresentados na interface de visualização.

\begin{figure}[!htb]
\centering
\includegraphics[width=300pt]{relatorio_overview.png}
\caption{Visão geral da interface.}
\end{figure}

Esta interface contém a informação sobre as duas estruturas de dados presentes no sistema, tabelas que contêm o histórico de vários eventos no sistema, e botões para interagir com o sistema e a visualização.

% TODO: Pôr uma introdução aqui.

\subsection*{Representação das Estruturas de Dados}
Na parte central da interface está apresentado um grafo, no qual os círculos representam os \emph{Nodes} e as setas representam os \emph{Links}. 
Também está presente uma etiqueta que indica qual o tipo do \emph{Node} corresponde à cor de um \emph{Node}.
As ligações (\emph{Links}) entre os \emph{Nodes} e as cores dos \emph{Nodes} são atualizadas dependendo do estado conhecido do sistema. 

Ou seja, há uma animação que vai evoluindo conforme o estado da rede conhecido pelo \emph{Node} de visualização, na qual é possível acompanhar que o comportamento de cada \emph{Node} altera a estado do rede.

É possível saber mais informação sobre cada \emph{Node}, deixando o ponteiro do rato por cima de um \emph{Node} e é possível forçar um \emph{Node} a realizar um pedido ao clicar neste.

Existe também o modo de filas/\emph{Queue}, que após pressionar o botão \emph{Queue Mode} (ver subsecção \ref{validacao:subsec:historico}
) as setas entre nós passam a demonstrar as ligações de filas/\emph{Queues} entre os \emph{Nodes}, ao invés das ligações do diretório.

Também estão apresentadas duas tabelas ao lado direito, com os títulos \emph{Queue} e \emph{Owner}, que representam a \emph{Queue} ``Principal'' e o atual \emph{Owner} conhecido.

\subsection*{Históricos}
\label{validacao:subsec:historico}
Na parte inferior da interface estão apresentadas 3 tabelas que contém informação sobre os seguintes históricos:
\begin{description}
    \item [\emph{Request History}] Histórico dos pedidos realizados pelos \emph{Nodes}.
    \item [\emph{Owner History}] Histórico dos \emph{Nodes} que tiveram o acesso ao objeto.
    \item [\emph{Queue History}] Histórico dos \emph{Nodes} da fila principal, isto é, a ordem da futura chegada do objeto aos \emph{Nodes}.
\end{description}

Esta informação não representa o estado da rede, mas é utilizada em testes. No entanto, o seu uso será descrito na secção \ref{validacao:sec:testes}.
% o grafo que representa o grafo distribuído/diretório e uma tabela que representa a fila/lista de espera ``Principal'' (que está diretamente ligada ao atual \emph{Owner}), 

\subsection*{Botões}
Na interface estão disponíveis 5 botões diferentes que permitem a iteração com o sistema e a visualização. A funcionalidade de cada botão é a seguinte:
\begin{description}
    \item [Remote Request All] - Força todos os \emph{Nodes} conhecidos a realizarem pedidos. Usado para testes.
    \item [Freeze] - Para qualquer atualização da interface de visualização.
    \item [Clear History] - Apaga as tabelas de histórico.
    \item [Toggle Movement] - Bloqueia o movimento de todos os círculos que representam os \emph{Nodes} no grafo.
    \item [Queue Mode] - Ativa o modo de demonstração das filas.

\end{description}

\section{Testes}
\label{validacao:sec:testes}
%TODO: explicar o quão diferente é a única implementação conhecida
No desenvolvimento foi necessário testar a implementação e também provar o seu bom funcionamento, no entanto não foi possível fazer-se uso de métodos formais de prova deste sistema, nem de comparar esta implementação com outra existente, visto que, a implementação feita no decorrer deste projeto é muito diferente da única conhecida.
						%mudar "histórico VVVVV"
Ainda assim é possível testar o bom funcionamento do sistema fazendo o uso dos elementos presentes na interface gráfica disponibilizada.
% Talvez especificar todas as questões de bom funcionamento?

Nem todos os elementos da visualização estão atualizados em qualquer momento, por exemplo, ambos o grafo e as tabelas da fila ``Principal'' podem estar em desacordo com o estado real do diretório, pois ambos dependem da última informação conhecida pelo \emph{Node} de visualização, que pode ter um \emph{Delay} causado tanto pela atualização (ou \emph{Refresh Rate}) da interface, a rede usada para comunicação, a sincronização da chegada de informação, etc., mas ao longo do decorrer do sistema, estes serão corrigidos.


É possível, na demonstração gráfica do diretório (o grafo) ser mostrado um estado impossível, como, por exemplo, mostrar vários \emph{Owners}, sendo que na realidade só existe um único, ou uma ligação em falta, porém, a implementação estaria errada caso um \emph{Node} se ligar a um \emph{Node} que previamente não era seu vizinho, pois, como definido no protocolo estudado, os vizinhos de um \emph{Node} são sempre as mesmos, apenas as ligações entre estes é que são alteradas (invertidas).

No entanto, o uso mais indicado do grafo e das tabelas da fila e atual \emph{Owner} é de visualização e demonstração de como funciona o sistema/protocolo, e não como prova, devido aos problemas de atualização presentes. 

Para uma evidência mais forte, fez-se uso das 3 tabelas dos históricos. 
Como referido anteriormente, a tabela \emph{Queue History} mostra o histórico dos \emph{Nodes} que entraram na fila ``Principal'', e a passagem do acesso ao objeto tem de seguir a ordem dessa fila, e a ordem por onde circula o objeto é mostrado na tabela \emph{Owner History}, logo, caso as tabelas \emph{Queue History} e \emph{Owner History} demonstrem os \emph{Nodes} com ordens diferentes, o sistema está errado.

A tabela \emph{Request History} demonstra que a ordem pela qual os \emph{Nodes} realizam pedidos ou que a ordem que \emph{Node} de visualização recebe a atualização de que esses \emph{Nodes} realizaram pedidos pode ser diferente da ordem na file ``Principal''.



\section{Conclusões}
\label{validacao:sec:conclusoes}
Neste capítulo foi feita uma descrição do funcionamento do sistema usado para a visualização e teste do programa \emph{Node}, sendo este um dos pontos de maior interesse deste projeto. O uso de ferramentas de construção provou-se útil na execução do sistema num contexto distribuído, de forma fácil, rápida, que pode ser usada para replicar a construção deste sem ter conhecimento da implementação do programa. Foi também uma explicação do funcionamento da interface e como se testou o sistema e demonstrou o bom funcionamento deste.
