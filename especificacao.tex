\chapter{Especificação}
\label{chap:especificacao}

\section{Introdução}
\label{especificacao:sec:intro}
\section{\emph{Node}}
Neste capítulo serão descritos os diferentes \emph{Nodes} presentes no diretório, tais como os seus atributos, 
que estão definidos em detalhe no capítulo seguinte \ref{especificacao:atr:section}.

Existem vários fatores que diferenciam os \emph{Nodes}: 
\begin{itemize}
    \item Detém o acesso ao objeto. - \emph{Owner}/Dono do objeto.
    \item Efetuou um pedido de aquisição do acesso ao objeto. - \emph{Waiter}/\emph{Node} em espera.
    \item Tem um pedido em espera. - \emph{With Request}/Com Pedido.
    \item Nenhum dos anteriores. - \emph{Idle}/Inativo.
\end{itemize}



Algumas regras de funcionamento do diretório incluem:

\begin{itemize}
    \item As ligações entre \emph{Nodes} são realizadas através do \emph{Link} de um \emph{Node}
    que aponta para o \emph{Channel \textbf{Find}} de um outro \emph{Node}.
    \item Cada pedido transmitido (através do \emph{Link}) inclui o \emph{Channel \textbf{Find}} do transmissor,   
    para que possa haver uma inversão na ligação, ou seja, para que o recetor possa atualizar 
    o seu \emph{Link} para o \emph{Channel \textbf{Find}} do transmissor.
    \item Caso o \emph{Node} detenha ou espera pelo acesso ao objeto e receba um pedido,
    este atualiza o \emph{WaiterChan} que aponta para o \emph{MyChan} do \emph{Node} que fez o pedido.
    \item Caso o \emph{Node} ceda o acesso ao objeto, este é transmitido através do \emph{WaiterChan} para o \emph{Node} em espera.
    \item Em qualquer momento/estado do diretório existe apenas um \emph{Owner}.
\end{itemize}


\subsubsection*{Outras definições}
% Esta secção não deve ficar bem.
Considera-se que existe um arco/uma ligação de um \emph{Node} \textbf{X} para um \emph{Node} \textbf{Y} quando o \emph{Link} do 
\emph{Node} \textbf{X}  é (ou aponta para) o \emph{Find} do \emph{Node} \textbf{Y}.
\\

Um \emph{Node} \textbf{X} tem a possibilidade de ter outro \emph{Node} \textbf{Y} (ou o pedido deste) em espera. Isto significa que o \emph{WaiterChan} do 
\emph{Node} \textbf{X} é (ou aponta para) o \emph{MyChan} do \emph{Node} \textbf{Y}.\\

O \emph{Child Node} de um \emph{Node} é o \emph{Node} que detém o \emph{Find} para qual o \textbf{Link} do \emph{Node} aponta. 
Isto é, existe um arco do \emph{Node} para o \emph{Child Node}, em que o \emph{Child Node} é o sucessor direto do \emph{Node}.
Cada \emph{Node} no diretório tem no máximo 1 \emph{Child Node}. \\

Os \emph{Parent Nodes} de um \emph{Node} são os \emph{Nodes} cujo o seu \textbf{Link} aponta para o \textbf{Find} do \emph{Node}.
Isto é, existe um arco de cada \emph{Parent Node} para o \emph{Node}, em que o \emph{Node} é o sucessor direto dos \emph{Parent Nodes}.
Não há qualquer limite de \emph{Parent Nodes}. \\

Não existem arcos/ligações de \emph{Nodes} para \emph{Nodes Idle}, pois as ligações apontam para \emph{Nodes} detentores ou futuros
detentores do acesso ao objeto (\emph{Waiter} e \emph{Owner}). \\







Serão descritos os comportamentos que cada tipo pode manifestar e transformações que estes podem sofrer, como a atualização de atríbutos e mudança de tipo:


%-------------------------
\subsection{\emph{Owner Terminal}}
\label{especificacao:nodes:owner_terminal}


\emph{Node} que detém o objeto e não tem pedido em espera.

Atributos:
\begin{itemize}
    \item Find
    \item MyChan
    \item Obj
\end{itemize}


\subsubsection*{Receção de um pedido Access Request}
O \emph{Node} recebe um pedido \emph{\textbf{Access Request}} no seu \emph{Channel \textbf{Find}},
que foi remetido pelo seu \emph{Parent Node}. 


Como o \emph{Node} é o detentor do acesso ao objeto, este transforma-se em \emph{\textbf{Owner With Request}},
isto é, é o detentor do acesso objeto mas existe um \emph{Node} que espera pelo acesso.
Atualiza o \textbf{\emph{Link}} (para \textbf{\emph{NewLink}}),
que aponta para o \textbf{Find} do seu \emph{Parent Node}, havendo uma inversão da ligação entre o \emph{Node} e o \emph{Parent Node},
e passa a deter o \textbf{\emph{WaiterChan}} que aponta para o \emph{MyChan} do \emph{Node} que realizou o pedido, isto para
que seja possível o envio do acesso ao objeto.


O \emph{Node} sofre a transformação \textbf{OwnerWithRequest(Find, MyChan, Obj, \underline{NewLink}, \underline{WaiterChan})}.



%-------------------------
\subsection{\emph{Owner With Request}}
\label{especificacao:nodes:owner_with_request}

\emph{Node} que detém o objeto e tem um pedido em espera.

Atributos:
\begin{itemize}
    \item Find
    \item MyChan
    \item Link
    \item Obj
    \item WaiterChan
\end{itemize}

\subsubsection*{Receção de um pedido Access Request}
O \emph{Node} recebe um pedido \emph{\textbf{Access Request}} no seu \emph{Channel \textbf{Find}},
que foi remetido pelo seu \emph{Parent Node}.

Este envia pelo \textbf{\emph{Link}} o \textbf{\emph{WaiterChan}} do pedido \emph{\textbf{Access Request}} e o seu \emph{Channel \textbf{Find}}.
É enviado o ``Find'' para que o seu \emph{Child Node} tenha a possibilidade de inverter a sua ligação, ou seja, para que possa haver um arco do \emph{Child Node} para o \emph{Node}, 
e o ``WaiterChan'' para que se faça chegar o \emph{MyChan} do \emph{Node} que fez o pedido a um \emph{Node} \emph{Waiter} ou \emph{Owner}.

Como o \emph{Node} já tem em espera um pedido de acesso, este mantém-se como \emph{\textbf{Owner With Request}},
mas atualiza o \textbf{\emph{Link}} (para \textbf{\emph{NewLink}}),
que aponta para o \textbf{Find} do seu \emph{Parent Node}, havendo uma inversão da ligação.

O \emph{Node} sofre a transformação \textbf{OwnerWithRequest(Find, MyChan, Obj, \underline{NewLink}, WaiterChan)}.


\subsubsection*{Cedência do Objeto}
Após a receção de um pedido \emph{\textbf{Access Request}}, o \emph{Node} pode ceder o acesso ao objeto ao \emph{Node} que fez o pedido.
Para tal, este envia pelo \emph{Channel \textbf{WaiterChan}} (O \emph{MyChan} do \emph{Node} que fez o pedido) um \emph{Channel \textbf{Give Access}}.

Como o node não detém o objeto e satisfez o pedido, este transforma-se em \emph{\textbf{Idle}} - \textbf{Idle(Find, MyChan, Link)}, em que apenas
mantém o \emph{Find}, o \emph{MyChan} e o \emph{Link}.





%-------------------------
\subsection{\emph{Idle}}
\label{especificacao:nodes:idle}

\emph{Node} não detém o objeto, não tem pedidos em espera e nem fez qualquer pedido. 

Atributos:
\begin{itemize}
    \item Find
    \item MyChan
    \item Link
\end{itemize}

\subsubsection*{Receção de um pedido Access Request}
O \emph{Node} recebe um pedido \emph{\textbf{Access Request}} no seu \emph{Channel \textbf{Find}}, que foi remetido pelo seu \emph{Parent Node}.

Este envia pelo \textbf{\emph{Link}} o \textbf{\emph{WaiterChan}} do pedido \emph{\textbf{Access Request}} e o seu \emph{Channel \textbf{Find}}.a
É enviado o ``Find'' para que o seu \emph{Child Node} tenha a possibilidade de inverter a sua ligação, ou seja, para que possa haver um arco do \emph{Child Node} para o \emph{Node}, 
e o ``WaiterChan'' para que se faça chegar o \emph{MyChan} do \emph{Node} que fez o pedido a um \emph{Node} \emph{Waiter} ou \emph{Owner}.

Como o \emph{Node} não tem o acesso ao objeto, este mantém-se como \emph{\textbf{Idle}}, e atualiza o \textbf{\emph{Link}} (para \textbf{\emph{NewLink}}), que aponta para o \textbf{Find} do seu \emph{Parent Node}.
\textbf{Idle(find, MyChan, Link)}.



\subsubsection*{Realização de um pedido de acesso}
O \emph{Node} envia no \textbf{\emph{Link}} o \textbf{\emph{MyChan}} e o \textbf{\emph{Find}} para o \emph{Child Node}.
O \emph{MyChan} para que seja possível chegar o acesso ao objeto a este \emph{Node}, e o \emph{Link} para que o seu \emph{Child Node} possa inverter a ligação.

Como fez um pedido, este transforma-se em \emph{\textbf{Waiter Terminal}}, e deixa de apresentar o \textbf{\emph{Link}} - \textbf{WaiterTerminal(Find, MyChan)}.

%-------------------------
\subsection{\emph{Waiter Terminal}}
\label{especificacao:nodes:waiter_terminal}

\emph{Node} aguarda pelo acesso ao objeto.

Atributos:
\begin{itemize}
    \item Find
    \item MyChan
\end{itemize}

\subsubsection*{Receção de um pedido Access Request}
O \emph{Node} recebe um pedido \emph{\textbf{Access Request}} no seu \emph{Channel \textbf{Find}}, que foi remetido pelo seu \emph{Parent Node}.

Este envia pelo \textbf{\emph{Link}} o \textbf{\emph{WaiterChan}} do pedido \emph{\textbf{Access Request}} e o seu \emph{Channel \textbf{Find}}.

É enviado o ``Find'' para que o seu \emph{Child Node} tenha a possibilidade de inverter a sua ligação, ou seja, para que possa haver um arco do \emph{Child Node} para o \emph{Node}, 
e o ``WaiterChan'' para que se faça chegar o \emph{MyChan} do \emph{Node} que fez o pedido a um \emph{Node} \emph{Waiter} ou \emph{Owner}.


Como o \emph{Node} não tem o acesso ao objeto mas aguarda pelo acesso ao objeto, este transforma-se em \emph{\textbf{Waiter With Request}}, atualiza o \textbf{\emph{Link}} (para \textbf{\emph{NewLink}}),
que aponta para o \textbf{Find} do seu \emph{Parent Node}, 
e passa a deter o \textbf{\emph{WaiterChan}}, que aponta para o \emph{MyChan} do \emph{Node} que realizou o pedido, isto para que seja possível o envio do acesso ao objeto.
Como o \emph{Node} passa a ter um pedido em espera, este sofre a transformação \textbf{WaiterWithRequest(Find, MyChan, \underline{NewLink}, \underline{NewWaiterChan})}.


\subsubsection*{Receção acesso ao objeto}
O \emph{Node} recebe acesso ao objeto (\textbf{Obj}) no seu \emph{Channel \textbf{MyChan}}.
Como o \emph{Node} não tem pedidos, este transforma-se em \textbf{\emph{Owner Terminal}}.
Como o \emph{Node} é detentor do objeto, deixam de existir ligações a partir do \emph{Node}, sofrendo a transformação \textbf{ OwnerTerminal(Find, MyChan, \underline{Obj}) }.




%-------------------------
\subsection{\emph{Waiter With Request}}
\label{especificacao:nodes:waiter_with_request}

\emph{Node} aguarda pelo acesso ao objeto e tem um pedido em espera.


Atributos:
\begin{itemize}
    \item Find 
    \item MyChan 
    \item Link 
    \item WaiterChan 
\end{itemize}



\subsubsection*{Receção de um pedido Access Request}
O \emph{Node} recebe um pedido \emph{\textbf{Access Request}} no seu \emph{Channel \textbf{Find}}, que foi remetido pelo seu \emph{Parent Node}.

Este envia pelo \textbf{\emph{Link}} o \textbf{\emph{WaiterChan}} do pedido \emph{\textbf{Access Request}} e o seu \emph{Channel \textbf{Find}}.

É enviado o ``Find'' para que o seu \emph{Child Node} tenha a possibilidade de inverter a sua ligação, ou seja, para que possa haver um arco do \emph{Child Node} para o \emph{Node}, 
e o ``WaiterChan'' para que se faça chegar o \emph{MyChan} do \emph{Node} que fez o pedido a um \emph{Node} \emph{Waiter} ou \emph{Owner}.

Como o \emph{Node} não tem o acesso ao objeto, aguarda pelo acesso ao objeto e tem um pedido em espera, este mantém-se como \emph{\textbf{Waiter With Request}},
porque ainda não satisfez o pedido que tem em espera, 
e atualiza o \textbf{\emph{Link}} (para \textbf{\emph{NewLink}}), que aponta para o \textbf{Find} do seu \emph{Parent Node}
Sofre a transformação \textbf{WaiterWithRequest(Find, MyChan, \underline{NewLink}, WaiterChan)}.


\subsubsection*{Receção acesso ao objeto}
O \emph{Node} recebe acesso ao objeto (\textbf{Obj}) no seu \emph{Channel \textbf{MyChan}}.
Como o \emph{Node} tem pedidos, este transforma-se em \textbf{\emph{Owner With Request}}, sendo a transformação \textbf{OwnerWithRequest(Find, MyChan, Obj, Link, WaiterChan)}.



\section{Atributos do \emph{Node}}
\label{especificacao:atr:section}
% Pouca informação
Neste secção serão descritos os atributos que podem constituir um \emph{Node}.

\subsection*{Find}
\label{especificacao:atr:Find}
    Este atributo representa o \emph{Channel} do \emph{Node} onde este recebe pedidos de acesso ao objeto.
    Está presente em todos os \emph{Nodes}.

\subsection*{MyChan}
\label{especificacao:atr:mychan}
    Este atributo representa o \emph{Channel} do \emph{Node} para o qual é transmitido o objeto.
    Está presente em todos os \emph{Nodes}.

\subsection*{Link}
\label{especificacao:atr:link}
    Este atributo representa a ligação do \emph{Node} para o seu \emph{Child Node}.
    Contém o \emph{Channel Find} do \emph{Child Node}.


\subsection*{Obj}
\label{especificacao:atr:obj}
    Este atributo representa o acesso ao objeto por parte do \emph{Node}.
    Em qualquer estado da rede, apenas um \emph{Node} dispões deste atributo.

\subsection*{WaiterChan}
\label{especificacao:atr:waiterchan}
    Este atributo representa o \emph{Channel} do \emph{Node} sucessor da fila.
    Contém o \emph{MyChan} do \emph{Node} em espera.


\section{\emph{Channels}}
\label{especificacao:sec:Channels}
Nesta especificação foram apenas definidos dois \emph{Channels}. A comunicação entre os \emph{Nodes} é feita por canais, pelos quais são comunicados canais.
Neste capítulo serão descritos os \emph{Channels}:


\subsection*{Access Request}
Este tipo de \emph{Channel} é usado para fazer chegar o pedido ao último elemento da fila de espera (de acesso ao objeto). 
Para tal, neste \emph{Channel} são comunicados dois \emph{Channels}:
\begin{itemize}
    \item O \emph{Channel \textbf{MyChan}} do \emph{Node} que fez o pedido.
    \item O \emph{Channel} que identifica quem fez chegar o pedido, ou seja, o \emph{\textbf{Find}} do \textbf{Parent Node}.
\end{itemize}


\subsection*{Give Access}
No entanto, há um tipo de \emph{Channel} que é usado para dar acesso ao objeto a quem fez o pedido, por outras palavras, é usado pelo atual \emph{\textbf{Owner}} para transmitir o acesso ao \emph{Waiter} que estava na posição da fila de espera.

\section{Conclusões}
