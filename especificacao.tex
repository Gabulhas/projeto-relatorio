\chapter{Especificação}
\label{chap:especificacao}

\section{Introdução}
\label{especificacao:sec:intro}
\section{\emph{Node}}

\begin{comment}
\end{comment}


Neste diretório existe apenas um \emph{Node} que detém o acesso ao objeto, pois um dos propósitos deste protocolo é garantir o acesso exclusivo a este. 


No entanto podem existir outros \emph{Nodes} que pretendem ter acesso ao objeto, e que ficam à espera deste. 

Também é possível que \emph{Nodes} detenham ou que esperam pelo acesso ao objeto, recebam um pedido de acesso de outro \emph{Node}, ou seja, os \emph{Nodes} podem ter pedidos em espera.

Ou seja, é possível que um \emph{Node} detenha o acesso ao objeto ou futuramente deterá e ter um outro \emph{Node} à espera que este ceda o acesso.

Estes fatores diferenciam os vários \emph{Nodes} no diretório,
e de seguida será feita uma enumeração destes fatores, sendo que estes podem coexistir:

\begin{itemize}
    \item Se o \emph{Node} detém o acesso ao objeto - \textbf{\emph{Owner}/Dono}
    \item Efetuou um pedido acesso ao objeto. - \textbf{\emph{Waiter}/\emph{Node} em espera. }

    \item Não tem pedidos em espera. - \textbf{\emph{Terminal}}
    \item Tem um pedido em espera. - \textbf{\emph{With Request}/Com Pedido.}

    \item Nenhum dos anteriores. - \textbf{\emph{Idle}/Inativo.}
\end{itemize}


A partir da combinação destes fatores temos os seguintes tipos de \emph{Nodes}:

\begin{description}
    \item [Owner Terminal] - \\ \emph{Node} que detém o acesso ao objeto e não tem pedidos em espera.
    \item [Owner With Request] - \\ \emph{Node} que detém o acesso ao objeto mas tem um pedido em espera.
    \item [Waiter Terminal] - \\ \emph{Node} que efetuou um pedido de acesso ao objeto mas não tem pedidos em espera.
    \item [Waiter With Request] - \\ \emph{Node} que efetuou um pedido de acesso ao objeto mas tem um pedido em espera.
    \item [Idle] - \\ \emph{Node} que não detém o acesso ao objeto nem efetuou pedido de acesso.
\end{description}



\section{Ligações}

Os \emph{Nodes} têm a possibilidade de comunicar entre si, quer usando o diretório, quando estes pretendem pedir o acesso ao objeto, ou diretamente, quando estes transmitem o acesso ao objeto para outro \emph{Node}.

As ligações (ou comunicações) entre os \emph{Nodes} são feitas a partir de \emph{Channels} (canais). Estes \emph{Channels} funcionam como caixa de entrada ou endereços dos \emph{Nodes}, em que podem ser enviados para outros \emph{Nodes}, e que, caso um \emph{Node} decide transmitir alguma mensagem para o outro \emph{Node}, esta transmissão é feita usando um \emph{Channel} do recetor da mensagem.

% provavelmente remover
Uma outra forma de definir o uso dos \emph{Channels} seria que este é formado por duas partes, a entrada e a saída, em que a entrada é usada pelo \emph{Node} transmissor e a saída está no \emph{Node} recetor.

Os \emph{Channels} são usados quando um \emph{Node}:
\begin{itemize}
    \item Pretende enviar uma mensagem para outro \emph{Node}, este fá-lo usando o \emph{Channel} pertencente ao recetor.
    \item Pretende receber mensagens de outros \emph{Nodes} este envia o seu \emph{Channel}.
\end{itemize}


\subsection*{Ligações no diretório}

Como referido no capítulo anterior, o diretório é um grafo (mais especificamente uma árvore de extensão mínima), em que os vértices são os \emph{Nodes} e as arestas são as ligações, e cada ligação é um conjunto de uma entrada (no transmissor) e uma saída (no recetor), o que indica que este grafo é dirigido, visto que há uma direção pela qual cada mensagem é transmitida por um \emph{Channel} (da entrada para a saída).

Neste tipo de ligação no diretório foram definidas duas partes nas ligações, o ``Link'', parte de entrada da ligação que está presente no transmissor, que se liga a um ``Find'' de um \emph{Node}, esta sendo a parte da saída das mensagens, que está presente no recetor.

Então, no grafo que representa o diretório, considera-se que existe uma ligação/um arco de um \emph{Node} \textbf{X} para um \emph{Node} \textbf{Y} quando o ``Link'' do 
\emph{Node} \textbf{X}  é (ou aponta para) o ``Find'' do \emph{Node} \textbf{Y}.

Então o \emph{Node} \textbf{Y} é o \emph{Child Node} do \textbf{X}, e o \emph{Node} \textbf{X} é o \emph{Parent Node} do \textbf{Y}.

Estas ligações do diretório são usadas apenas para a transmissão e circulação dos pedidos de acesso ao objeto, visto que um \emph{Node} não tem informação da localização deste mas tem informação sobre a ligação para o seu \emph{Child Node},
e o conjunto das ligações aponta sempre para a localização atual (o \emph{Owner}) ou uma futura localização (um \emph{Waiter}), ou seja, o \emph{Node} não tem informação sobre a localização atual ou uma futura, mas o conjunto de todas as ligações forma essa informação.

Estes dois tópicos serão retratados em subcapítulos seguintes, mas é necessário mencioná-los:
\begin{itemize}
    \item Quando um \emph{Node} recebe um pedido de acesso, a ligação entre este e o \emph{Node} que lhe transmitiu é invertida. Esta inversão é desempenhada para que as ligações dos \emph{Nodes} tenham sempre a direção para onde está ou estará o objeto.

    \item Os pedidos de acesso contém informação necessária para a inversão das ligações e para que seja possível o envio do objeto do atual \emph{Owner} para o \emph{Node} que realizou o pedido.
\end{itemize}

% para que servem
% Quais é que são os canais
% Qual é a entrada e a saída
% o que é transmitido

\subsection*{Ligação direta}
É possível que um \emph{Node} tenha outro \emph{Node} à espera que lhe envie o acesso ao objeto. No entanto para que este tenha um pedido em espera é necessário que este seja o detentor do objeto ou que futuramente tenha o acesso (\emph{Waiter}).

Quando o \emph{Node} tem outro \emph{Node} em espera, há uma ligação direta entre estes.
Esta ligação é utilizada no envio do acesso ao objeto, pois o \emph{Node} quando receber o acesso, este poderá ceder ao \emph{Node} que espera através desta ligação.

Neste tipo de ligação são definidas duas partes, o ``WaiterChan'', parte de entrada da ligação que está presente no transmissor, que se liga ao ``MyChan'' de um \emph{Node}, sendo esta a parte da saída do acesso ao objeto, que está presente no recetor.

Com a mudança das ligações no diretório e o facto de que um \emph{Node} pode ter um (e apenas um) outro \emph{Node} à espera, isto resulta na circulação de um pedido até este chegar a um \emph{Node} que detém ou futuramente deterá (mas que espera) o acesso ao objeto e que não tenha qualquer outro pedido.

Neste sistema é possível existirem \emph{Nodes} à espera de outros \emph{Nodes} que também estão à espera, e cada um tem no máximo um outro \emph{Node} à espera. 
Este comportamento dos \emph{Nodes} origina a criação de uma fila (\emph{Queue}), pois o \emph{Node} que espera pelo atual detentor do objeto irá ser o primeiro da fila e posteriormente enviará o acesso ao objeto ao \emph{Node} que espera por este, e assim seguidamente.

A cabeça desta fila será o atual detentor do objeto, que está diretamente ligado a um outro \emph{Node}, e o último elemento da fila será o \emph{Node} na fila que não tem qualquer outro \emph{Node} à sua espera.




\section{Atributos do \emph{Node}}
\label{especificacao:atr:section}
Neste secção serão descritos os atributos que podem constituir um \emph{Node}.

\subsection*{Find}
\label{especificacao:atr:Find}
    Este atributo representa o \emph{Channel} do \emph{Node} onde este recebe pedidos de acesso ao objeto.
    Está presente em todos os \emph{Nodes}.

\subsection*{MyChan}
\label{especificacao:atr:mychan}
    Este atributo representa o \emph{Channel} do \emph{Node} para o qual é transmitido o objeto.
    Está presente em todos os \emph{Nodes}.

\subsection*{Link}
\label{especificacao:atr:link}
Este atributo representa a ligação do \emph{Node} para o ``Find'' de um outro \emph{Node}, ou seja, uma ligação para o seu \emph{Child Node}.

\subsection*{WaiterChan}
\label{especificacao:atr:waiterchan}
    Este atributo representa o \emph{Channel} para o ``MyChan'' de um outro \emph{Node}, ou seja, o seu sucessor na fila.

\subsection*{Obj}
\label{especificacao:atr:obj}
    Este atributo representa o acesso ao objeto por parte do \emph{Node}.
    Em qualquer estado da rede, apenas um \emph{Node} dispões deste atributo.


\subsection*{Atributos de cada tipo de \emph{Node}}
Neste subcapítulo serão definidos os atributos que cada tipo de \emph{Node} apresenta.


\subsubsection*{\emph{Owner Terminal}}
\begin{itemize}
    \item Find
    \item MyChan
    \item Obj
\end{itemize}
\subsubsection*{\emph{Owner With Request}}
\begin{itemize}
    \item Find
    \item MyChan
    \item Link
    \item Obj
    \item WaiterChan
\end{itemize}
\subsubsection*{\emph{Idle}}
\begin{itemize}
    \item Find
    \item MyChan
    \item Link
\end{itemize}
\subsubsection*{\emph{Waiter Terminal}}
\begin{itemize}
    \item Find
    \item MyChan
\end{itemize}
\subsubsection*{\emph{Waiter With Request}}
\begin{itemize}
    \item Find 
    \item MyChan 
    \item Link 
    \item WaiterChan 
\end{itemize}



\section{Tipos de mensagens}
\label{especificacao:sec:Channels}
Nesta especificação foram apenas definidos dois \emph{Channels}. A comunicação entre os \emph{Nodes} é feita por canais, pelos quais são comunicados canais (as mensagens).


\subsection*{Access Request}
Este tipo de \emph{Channel} é usado para fazer chegar o pedido a um \emph{Node} que detém ou futuramente irá deter o acesso ao objeto. O \emph{Node} de destino depende da estrutura atual do diretório, no entanto o \emph{Node} que realizou o pedido não terá informação sobre o destino, mas o recetor de destino terá a informação necessária para que este tenha a possibilidade de ceder o acesso ao objeto ao \emph{Node} que realizou o pedido.

Para tal, neste \emph{Channel} são comunicados dois \emph{Channels}:
\begin{itemize} \item O \emph{Channel \textbf{MyChan}} do \emph{Node} que fez o pedido.
    \item O \emph{Channel} que identifica quem fez chegar o pedido, ou seja, o \emph{\textbf{Find}} do \textbf{Parent Node}.
\end{itemize}


\subsection*{Give Access}
No entanto, há um tipo de \emph{Channel} que é usado para ceder o acesso ao objeto a quem fez o pedido, por outras palavras, é usado pelo atual \emph{\textbf{Owner}} para transmitir o acesso ao \emph{Waiter} que estava à sua espera, ou seja, ao primeiro da \emph{Queue}.



\section{Comportamentos dos \emph{Nodes}}
%-------------------------

Neste capítulo serão 

\subsection*{\emph{Owner Terminal}}
\label{especificacao:nodes:owner_terminal}


\subsubsection*{Receção de um pedido Access Request}
O \emph{Node} recebe um pedido \emph{\textbf{Access Request}} no seu \emph{Channel \textbf{Find}},
que foi remetido pelo seu \emph{Parent Node}. 


Como o \emph{Node} é o detentor do acesso ao objeto, este transforma-se em \emph{\textbf{Owner With Request}},
isto é, é o detentor do acesso objeto mas existe um \emph{Node} que espera pelo acesso.
Atualiza o \textbf{\emph{Link}} (para \textbf{\emph{NewLink}}),
que aponta para o \textbf{Find} do seu \emph{Parent Node}, havendo uma inversão da ligação entre o \emph{Node} e o \emph{Parent Node},
e passa a deter o \textbf{\emph{WaiterChan}} que aponta para o \emph{MyChan} do \emph{Node} que realizou o pedido, isto para
que seja possível o envio do acesso ao objeto.


O \emph{Node} sofre a transformação \textbf{OwnerWithRequest(Find, MyChan, Obj, \underline{NewLink}, \underline{WaiterChan})}.



%-------------------------
\subsection*{\emph{Owner With Request}}
\label{especificacao:nodes:owner_with_request}

\subsubsection*{Receção de um pedido Access Request}
O \emph{Node} recebe um pedido \emph{\textbf{Access Request}} no seu \emph{Channel \textbf{Find}},
que foi remetido pelo seu \emph{Parent Node}.

Este envia pelo \textbf{\emph{Link}} o \textbf{\emph{WaiterChan}} do pedido \emph{\textbf{Access Request}} e o seu \emph{Channel \textbf{Find}}.
É enviado o ``Find'' para que o seu \emph{Child Node} tenha a possibilidade de inverter a sua ligação, ou seja, para que possa haver um arco do \emph{Child Node} para o \emph{Node}, 
e o ``WaiterChan'' para que se faça chegar o \emph{MyChan} do \emph{Node} que fez o pedido a um \emph{Node} \emph{Waiter} ou \emph{Owner}.

Como o \emph{Node} já tem em espera um pedido de acesso, este mantém-se como \emph{\textbf{Owner With Request}},
mas atualiza o \textbf{\emph{Link}} (para \textbf{\emph{NewLink}}),
que aponta para o \textbf{Find} do seu \emph{Parent Node}, havendo uma inversão da ligação.

O \emph{Node} sofre a transformação \textbf{OwnerWithRequest(Find, MyChan, Obj, \underline{NewLink}, WaiterChan)}.


\subsubsection*{Cedência do Objeto}
Após a receção de um pedido \emph{\textbf{Access Request}}, o \emph{Node} pode ceder o acesso ao objeto ao \emph{Node} que fez o pedido.
Para tal, este envia pelo \emph{Channel \textbf{WaiterChan}} (O \emph{MyChan} do \emph{Node} que fez o pedido) um \emph{Channel \textbf{Give Access}}.

Como o node não detém o objeto e satisfez o pedido, este transforma-se em \emph{\textbf{Idle}} - \textbf{Idle(Find, MyChan, Link)}, em que apenas
mantém o \emph{Find}, o \emph{MyChan} e o \emph{Link}.





%-------------------------
\subsection*{\emph{Idle}}
\label{especificacao:nodes:idle}


\subsubsection*{Receção de um pedido Access Request}
O \emph{Node} recebe um pedido \emph{\textbf{Access Request}} no seu \emph{Channel \textbf{Find}}, que foi remetido pelo seu \emph{Parent Node}.

Este envia pelo \textbf{\emph{Link}} o \textbf{\emph{WaiterChan}} do pedido \emph{\textbf{Access Request}} e o seu \emph{Channel \textbf{Find}}.a
É enviado o ``Find'' para que o seu \emph{Child Node} tenha a possibilidade de inverter a sua ligação, ou seja, para que possa haver um arco do \emph{Child Node} para o \emph{Node}, 
e o ``WaiterChan'' para que se faça chegar o \emph{MyChan} do \emph{Node} que fez o pedido a um \emph{Node} \emph{Waiter} ou \emph{Owner}.

Como o \emph{Node} não tem o acesso ao objeto, este mantém-se como \emph{\textbf{Idle}}, e atualiza o \textbf{\emph{Link}} (para \textbf{\emph{NewLink}}), que aponta para o \textbf{Find} do seu \emph{Parent Node}.
\textbf{Idle(find, MyChan, Link)}.



\subsubsection*{Realização de um pedido de acesso}
O \emph{Node} envia no \textbf{\emph{Link}} o \textbf{\emph{MyChan}} e o \textbf{\emph{Find}} para o \emph{Child Node}.
O \emph{MyChan} para que seja possível chegar o acesso ao objeto a este \emph{Node}, e o \emph{Link} para que o seu \emph{Child Node} possa inverter a ligação.

Como fez um pedido, este transforma-se em \emph{\textbf{Waiter Terminal}}, e deixa de apresentar o \textbf{\emph{Link}} - \textbf{WaiterTerminal(Find, MyChan)}.

%-------------------------
\subsection*{\emph{Waiter Terminal}}
\label{especificacao:nodes:waiter_terminal}


\subsubsection*{Receção de um pedido Access Request}
O \emph{Node} recebe um pedido \emph{\textbf{Access Request}} no seu \emph{Channel \textbf{Find}}, que foi remetido pelo seu \emph{Parent Node}.

Este envia pelo \textbf{\emph{Link}} o \textbf{\emph{WaiterChan}} do pedido \emph{\textbf{Access Request}} e o seu \emph{Channel \textbf{Find}}.

É enviado o ``Find'' para que o seu \emph{Child Node} tenha a possibilidade de inverter a sua ligação, ou seja, para que possa haver um arco do \emph{Child Node} para o \emph{Node}, 
e o ``WaiterChan'' para que se faça chegar o \emph{MyChan} do \emph{Node} que fez o pedido a um \emph{Node} \emph{Waiter} ou \emph{Owner}.


Como o \emph{Node} não tem o acesso ao objeto mas aguarda pelo acesso ao objeto, este transforma-se em \emph{\textbf{Waiter With Request}}, atualiza o \textbf{\emph{Link}} (para \textbf{\emph{NewLink}}),
que aponta para o \textbf{Find} do seu \emph{Parent Node}, 
e passa a deter o \textbf{\emph{WaiterChan}}, que aponta para o \emph{MyChan} do \emph{Node} que realizou o pedido, isto para que seja possível o envio do acesso ao objeto.
Como o \emph{Node} passa a ter um pedido em espera, este sofre a transformação \textbf{WaiterWithRequest(Find, MyChan, \underline{NewLink}, \underline{NewWaiterChan})}.


\subsubsection*{Receção acesso ao objeto}
O \emph{Node} recebe acesso ao objeto (\textbf{Obj}) no seu \emph{Channel \textbf{MyChan}}.
Como o \emph{Node} não tem pedidos, este transforma-se em \textbf{\emph{Owner Terminal}}.
Como o \emph{Node} é detentor do objeto, deixam de existir ligações a partir do \emph{Node}, sofrendo a transformação \textbf{ OwnerTerminal(Find, MyChan, \underline{Obj}) }.




%-------------------------
\subsection*{\emph{Waiter With Request}}
\label{especificacao:nodes:waiter_with_request}


\subsubsection*{Receção de um pedido Access Request}
O \emph{Node} recebe um pedido \emph{\textbf{Access Request}} no seu \emph{Channel \textbf{Find}}, que foi remetido pelo seu \emph{Parent Node}.

Este envia pelo \textbf{\emph{Link}} o \textbf{\emph{WaiterChan}} do pedido \emph{\textbf{Access Request}} e o seu \emph{Channel \textbf{Find}}.

É enviado o ``Find'' para que o seu \emph{Child Node} tenha a possibilidade de inverter a sua ligação, ou seja, para que possa haver um arco do \emph{Child Node} para o \emph{Node}, 
e o ``WaiterChan'' para que se faça chegar o \emph{MyChan} do \emph{Node} que fez o pedido a um \emph{Node} \emph{Waiter} ou \emph{Owner}.

Como o \emph{Node} não tem o acesso ao objeto, aguarda pelo acesso ao objeto e tem um pedido em espera, este mantém-se como \emph{\textbf{Waiter With Request}},
porque ainda não satisfez o pedido que tem em espera, 
e atualiza o \textbf{\emph{Link}} (para \textbf{\emph{NewLink}}), que aponta para o \textbf{Find} do seu \emph{Parent Node}
Sofre a transformação \textbf{WaiterWithRequest(Find, MyChan, \underline{NewLink}, WaiterChan)}.


\subsubsection*{Receção acesso ao objeto}
O \emph{Node} recebe acesso ao objeto (\textbf{Obj}) no seu \emph{Channel \textbf{MyChan}}.
Como o \emph{Node} tem pedidos, este transforma-se em \textbf{\emph{Owner With Request}}, sendo a transformação \textbf{OwnerWithRequest(Find, MyChan, Obj, Link, WaiterChan)}.


\section{Conclusões}
Neste capítulo foram estudados todos os elementos que pertencem ao sistema,
tais como os seus comportamentos e utilizades.
A separação das várias propriedades destes elementos definidas na descrição serviu de auxílio para a implementação deste, com principal destaque o comportamento de cada tipo de nó e a mudança de estado após a despoletação deste.

