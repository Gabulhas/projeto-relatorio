% porque
\chapter{Motivação}
\label{chap:estado-da-arte}
% Porquê
% Distribuido
% que problemas
% Escalabilidade
% evitar bottlenecks
% 

\section{Introdução}
\label{chap2:sec:intro}

% explicar a utilidade/problemas resolvidos com este protocolo
%Qual é o interesse em fazer este trabalho

% explicar que problemas concorrentes e/ou distribuídos são resolvidos de forma diferente dos problemas sequenciais



%%%%%NOTAS%%%%%
% saber explicar as dificuldades da implementação do AADP (não a minha em específico, mas em geral), os seus usos e 
% porque é que é um problema relevante
% explicar benefícios da visualiazção (ponto importante, que até está no nome do AADP, as Setas)
% motivação do trabalho, porque é que é giro de se fazer
% estrutura da implementação
% que problemas estamos a resolver
  % definição do problema
  % a forma como vamos resolver os problemas
% objetivos
% quais os desafios/interesses ao fazer o projeto



\section{Trabalhos relacionados}
%Ivy e Arvy
% comparar este algoritmo com outros (ver http://cs.brown.edu/people/mph/DemmerH98/disc.pdf)


\section{Descrição do Protocolo}

% notas paper
% explicar o nome
% explicar que ao seguirmos as setas chegamos sempre ao objeto ou alguém que terá acesso ao mesmo
% explicar o que é um diretório
% dar uma vista global sobre
% - Objeto Movível
% O objeto move-se pelo diretório dependendo se houve pedidos por parte de Nodes



\subsection{Estrutura do Diretório}
\begin{comment}
     elementos
     - Nodes
    vértices do grafo
     

     - Ligações entre os nodes (ligação de finds e ligação do MyChan, ver ARVY)
     explicar porque é que as ligações viram
     - envios do obj são diretos
\end{comment}

\subsection{Características do Diretório}
% 4.1 do paper

\subsection{Estruturas de Dados}
%MST
%queue

\section{Conclusões}
\label{chap2:sec:concs}
