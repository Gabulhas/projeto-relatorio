% porque
\chapter{Motivação}
\label{chap:estado-da-arte}

\section{Introdução}
\label{chap2:sec:intro}


% explicar que o mundo cada vez mais requer programas distribuídos devido à escalabilidade destes
% aumento de cores no CPU
% aumento da infraestruturas das grandes empresas
% A internet é cada vez mais global, logo é necessário distribuir os sistemas por várias partes do mundo
Atendendo à crescente utilização de grandes infraestruturas, como redes sociais, grandes empresas e a demanda de maior capacidade de processamento, como por exemplo, o aumento do número de núcloes nos processadores, a necessidade do uso de sistemas distribuídos tem vindo a aumentar.
%https://www.oreilly.com/content/distributed-systems-a-quick-and-simple-definition/
Das várias vantagens no uso de sistemas distribuídos, as que mais se destacam são as seguintes.
% maior runtime
% fault tolearnce
% mais baratos porque escabilidade horizontal
% mais eficientes, porque é possivel várias máquinas trabalharem

% remover, demasiada informação desnecessária, não o ponto principal da motivação
\begin{description}
    \item [Escalabilidade Horizontal] Os sistemas distribuídos permitem que se aumente a capacidade de processamento e armazenamento ao introduzir máquinas no sistema, ao invés de melhorar uma única máquina.
	Exemplo: É mais dispendioso adquirir uma máquina de uma grande capacidade de processamento do que várias de uma capacidade menor para obter a mesma capacidade em conjunto.


    \item [Maior tolerância a erros]Quando o processamento está distribuído por vários nós, a falha de um único nó não levará a que o sistema num todo falhe.

	Exemplo: Em uma Base De Dados distribuída, caso um dos nós falhe e haja perda de informação, existem outras Bases de Dados (réplicas) que contém a mesma informação que esta.


    \item [Mais eficientes] O uso de algoritmos distribuídos permite um maior número de máquinas em execução concorrente, e consequentemente, a divisão do trabalho pelas várias máquinas.
	Também é possível a distribuição da localização física das máquinas, o que permite conexões de maior velocidade em outros locais no mundo.

	Por exemplo: Uma rede social, à qual se conectam milhões de utilizadores, seria provável que o servidor (ou monólito) não tivesse capacidade de processar todos os pedidos concorrentes de forma eficaz, logo o processamento de novos pedidos seria demorado. Também é benéfico que o sistema esteja distribuído pelo mundo para que independentemente da localização do utilizador, este tenha um acesso rápido à rede.



\end{description}

No entanto, os sistemas distribuídos, que inerentemente são concorrentes, têm uma complexidade maior no desenvolvimento em comparação com sistemas e programas sequênciais.
A falta de um relógio central, a possibilidade de vários processos necessitarem de aceder ao mesmo recurso ao mesmo tempo, ordem indeterminada de quais quer pedidos, a necessidade de se usar uma rede para comunicar informação entre os nós e dificuldade de controlar os vários sistemas independentes pertencentes ao sistema distribuídos são alguns dos fatores para esta complexidade.


\section{Descrição do Protocolo}

Nesta secção será descrito o funcionamento e estrutura do \textit{Arrow Distributed Directory Protocol} (\emph{Arrow}). 

Este sistema consiste numa rede, que permite o envio assíncrono de mensagens entre os nós, e um diretório, que permite localizar um objeto na rede e garantir o acesso exclusivo a este. 

O objeto é considerado ``móvel'', porque há a possibilidade de este se movimentar na rede entre os nós.
O objeto pode ser considerado um processo, um ficheiro ou qualquer outra estrutura de dados.

Cada nó tem uma só ligação, deste para outro nó, no entanto pode haver qualquer número de nós com ligação a um único.

Quando um nó pretende aceder ao objeto, este envia um pedido de acesso pela sua ligação e
caso um nó receba um pedido de acesso, a ligação deste passa a apontar para o nó vizinho que lhe fez chegar o pedido, ou seja, por onde passa o pedido há uma inversão do sentido da ligação.

Numa vista global sobre a rede, quando um nó realiza um pedido, as ligações dos nós por onde esse pedido passou passam a indicar onde é que futuramente estará o objeto.

Se um novo nó, que foi afetado pela passagem do pedido, realizar um pedido de acesso, este chegará ao nó que anteriormente fez o pedido.

Esta inversão das ligações permite que, o nó apenas detendo uma ligação, faça chegar o seu pedido ao nó que detém o objeto ou a um nó que deterá o objeto mas que espera por ele.

Caso um nó que espera pelo objeto, quer porque o seu pedido ainda está em circulação na rede ou o nó detentor do objeto ainda não o cedeu, receba um pedido de acesso de outro nó, então este armazena o uma ligação nó que realizou o pedido.

Quando o nó detém o objeto,se recebe um pedido ou tem em espera um outro nó, o objeto é cedido através de uma ligação direta entre dois nós, evitando a passagem do objeto pelo diretório.

A particularidade deste protocolo deve-se à mudança tanto da localização do objeto como a sua origem, pois esta muda-se para o nó que tem o acesso ao objeto, e não há um único nó que detém a localização atual do objeto, mas a disposição de todas as ligações permite a localização do objeto por todos os nós.

O facto da origem do objeto estar em constante alteração evita que apenas um nó detenha o acesso ao objeto, ou seja, o acesso exclusivo, e que e que nenhum nó se torne num foco, ou seja, que nenhum nó receba demasiados pedidos em relação a outros nós, provocado um engarrafamento/\emph{bottleneck} no acesso ao objeto.

% notas paper
% explicar o nome
% explicar que ao seguirmos as setas chegamos sempre ao objeto ou alguém que terá acesso ao mesmo
% explicar o que é um diretório
% dar uma vista global sobre
% - Objeto Movível
% O objeto move-se pelo diretório dependendo se houve pedidos por parte de Nodes

\subsection{Estrutura do Diretório}


O diretório constitui um grafo, mais especificamente uma árvore de extensão mínima, em que os vértices são os nós pertencentes ao diretório e as arestas as ligações entre os nós.


\begin{comment}
     elementos
     - Nodes
    vértices do grafo
     

     - Ligações entre os nodes (ligação de finds e ligação do MyChan, ver ARVY)
     explicar porque é que as ligações viram
     - envios do obj são diretos
\end{comment}

\subsection{Estruturas de Dados}
%MST
%queue


\section{Trabalhos relacionados}
%Ivy e Arvy
De muitos algoritmos distribuídos e suas implementações, há dois trabalhos muito relacionados com o protocolo tratado neste projeto. Estes são, o \emph{Ivy} \cite{Ivy} e o \emph{Arvy} \cite{Arvy}.
O \emph{Ivy} apresenta um funcionamento muito similar ao \emph{Arrow}, em que a origem do objeto muda com a posição atual deste, isto é, 


Na implementação do mesmo protocolo retratado neste projeto existe apenas...


% comparar este algoritmo com outros (ver http://cs.brown.edu/people/mph/DemmerH98/disc.pdf)

% saber explicar as dificuldades da implementação do AADP (não a minha em específico, mas em geral), os seus usos e 
% porque é que é um problema relevante
% explicar benefícios da visualiazção (ponto importante, que até está no nome do AADP, as Setas)
% motivação do trabalho, porque é que é giro de se fazer
% explicar a utilidade/problemas resolvidos com este protocolo
% que problemas estamos a resolver
  % definição do problema
  % a forma como vamos resolver os problemas
% Qual é o interesse em fazer este trabalho
% quais os desafios/interesses ao fazer o projeto

\section{Conclusões}
\label{chap2:sec:concs}
