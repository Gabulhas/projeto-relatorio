\chapter{Implementação}
\label{chap:imp}

\section{Introdução}
\label{chap4:sec:intro}


\section{Escolhas de Implementação}
\label{chap4:escolhas_implementacao}
% Explicar o uso das bibliotecas
% Pôr a visualização à parte
% Explicar o uso de pedidos HTTP (por ser um contexto distribuido, e HTTP ser um dos protocolos mais usados
% Explicar uso de Containers Docker

\section{Detalhes de Implementação}
% problemas da concorrência
% Explicar a divisão em Modulos e em ficheiros

\section{Classe \emph{Node}}
A Classe \emph{Node} desempenha a função de armazenar o estado atual do próprio \emph{Node}, define os métodos/procedimentos que este pode executar, e uma enumeração dos 5 diferentes tipos de \emph{Nodes}.

Nesta Implementação, esta classe está incluida num módulo \emph{Go} ``Nodes'' (Ou seja, num diretório com o mesmo nome), e é consituído por 5 ficheiros, sendo que as várias (funcionalidades ?) estão distribuídas por estes ficheiros.

Estes são:
\begin{itemize}
    \item Node.go - Contém \emph{Struct} que define os \textbf{atributos}.
    \item NodeBehaviours.go - Define os possíveis \textbf{comportamentos}.
    \item NodeCommunications.go - Conjunto de \textbf{métodos de comunicação} de informação para outros \emph{Nodes}.
    \item NodeTranformations.go - \textbf{Transformações}/Mudanças de tipo que o \emph{Node} pode sofrer.
    \item NodeType.go - Enumeração dos \textbf{tipos} que o \emph{Node} pode ser.
\end{itemize} 


\subsection{Atributos da Classe}
Como referido no capítulo de Especificação \ref{especificacao:atr:section}, um \emph{Node} tem, no máximo 5 atributos, no entanto, na sua implementação este incluí no total 8.
Esta é a definição da \emph{Struct} dos atributos do \emph{Node}.


\begin{lstlisting}[caption={Definição da estrutura \emph{Node}},language=Go]
type Node struct {
	Type       NodeType //Tipo do Node, ver Tipos de Nodes
	MyChan     string   //Channel onde recebe acesso ao objeto
	Find       string   //Channel onde recebe pedidos
	Link       string   //Ligação para o child Node
	WaiterChan string   //Channel do Node que está na posição seguinte da fila
	MyAddress  string   //Endereço do Node
	VisAddress string   //Endereço para onde envia o seu estado atual para a atualização da visualização
	Obj        bool     //Se tem objeto ou não 
}

\end{lslisting}
Na \emph{Struct} estão definidos todos os atributos que o \emph{Node} pode deter, porém, quando um atributo é ``inexistente'', este é definido como vazio, ou seja, os atributos ``WaiterChan'', ``VisAddress'' e ``Link'' podem ser \emph{strings} vazias.


\subsection{Comportamentos}
Como referido no capítulo de Especificação \ref{especificacao:atr:section},

\subsubsection{Receção de um pedido Access Request}
\subsubsection{Cedência do Objeto}
\subsubsection{Realização de um pedido de acesso}
\subsubsection{Receção acesso ao objeto}

\subsection{Tranformações dos \emph{Nodes}}
\subsection{Tipos de Nodes}
\label{chap:imp:node:tipos}


\section{Inicialização do \emph{``Self'' Node}}

\section{Comunicação entre Nodes}
\section{Classes de \emph{Channels}}



\section{Implementação da Visualização}
% como os dados são atualizados
% como os dados são passados para a página
% como os dados são demonstrados no grafo
% como a queue é formada
% como o histórico é formado

\section{Conclusões}
