\chapter{Conclusões e Trabalho Futuro}
\label{chap:conc-trab-futuro}

\section{Conclusões Principais}
\label{sec:conc-princ}

\begin{comment}
    Lições aprendidas
    - Implementaçã de um sistema concorrente e distribuído
    - Evitar race conditions
    - Oportunidade de usar Mecanismos de Sincronização e Concorrência

    Pontos Principais
    - o que é que foi possível fazer
    - o que é em geral este trabalho/implementação
    - possível implementação moderna com visualização do algoritmo 

    Neste trabalho foi possível estudar e implementar o sistema concorrente e distribuído, que faz uso do ``Arrow Distributed Directory Protocol''.
    O que torna notável este protocolo é o mecanismo de inversão ou troca de ligações, que mesmo sendo simples,
    evita que mais que um elemento no sistema tenha acesso a um objeto que pretendemos ter o acesso exclusivo, isto é, evita falhas de Condições de Corrida.
    Este tipo de falhas pode ter grande impacto no bom funcionamento de programas concorrentes, pois a leitura e/ou escrita concorrente de 
    informação pode levar a resultados imprevisíveis. 

    Após um estudo do o artigo que introduz o \acs{ADDP} \cite{Arrow}, que permitiu a realização de uma especificação dos elementos que pertencem ao sistema definido por o protocolo estudado,
    esta que foi usada como base no planeamento e implementação do projeto.
    No desenvolvimento do programa executado no sistema distribuído, foi possível praticar o uso de mecanismos de sincronização e concorrentes, estes oferecidos pela Linguagem \emph{Go},
    a linguagem escolhida para a implementação.
    Para além da implementação do programa distribuído, foi desenvolvida uma interface, que permitiu visualizar o protocolo em ação, testar a implementação, e provar (parcialmente) o seu
    bom funcionamento e fidelidade com o descrito na Especificação.
\end{comment}


Este projeto teve como objetivo principal o estudo e implementação do \acs{ADDP}, um protocolo que permite a partilha de um objeto móvel através de um diretório, 
sem que todos os elementos detenham a localização deste
e que garante o acesso mutuamente exclusivo a este. 
Num sistema distribuído definido por este protocolo, os vários nós que o formam têm a possibilidade de requisitar o acesso (exclusivo) a um objeto,
fazendo uso das ligações a outros nós, as quais são invertidas após serem usadas na transmissão dos pedidos. Este mecanismo de inversão garante que a direção de qualquer
ligação indique a atual ou futura localização do objeto (ou o acesso a este), evitando a necessidade de os nós difundirem a sua localização.

Esta característica garante uma maior escalabilidade do sistema, pois não é necessário que os nós transmitam a informação a todos os outros nós no sistema sobre a localização
do objeto, e garantindo o acesso exclusivo a este ao enviá-lo ao nó ao qual o primeiro pedido a ser processado pertence, fazendo com que os seguintes pedidos sejam redirecionados
ao futuro detentor.
As várias ligações entre os nós e a informação pelas quais é transferida definem duas estruturas de dados distribuídas, sendo estas um grafo, mais especificamente uma árvore de extensão mínima
formada pelas ligações de diretório onde são transmitidos os pedidos de acesso ao objeto, e uma fila formada pelas ligações exteriores ao diretório pelo qual circula o acesso ao objeto.

Após a leitura e estudo do artigo que introduz o \acs{ADDP} \cite{Arrow}, foi realizada uma especificação de todos os elementos que pertencem ao sistema, nomeadamente os vários nós, os seus atributos,
as ligações entre estes, os tipos de mensagem que por estas circulam e os comportamentos do nó, que serviu de base para a implementação. 

Foi realizada uma implementação na Linguagem \emph{Go} do protocolo, na qual foi possível explorar conceitos de concorrência, comunicação por canais, sincronização e exclusão mútua,
na qual foi possível obter um programa que é executado num sistema distribuído. 
Para a visualização deste sistema, foi desenvolvido o programa de um nó especial com o propósito de reconstruir o estado do sistema. 
Através de atualizações que cada nó envia periodicamente a este, este nó é capaz de reconstruir o grafo que representa o diretório, as filas existentes no sistema e, sem uso de \emph{timestamps},
reconstruir a ordem dos eventos ocorridos, usando regras do protocolo para obter a sequência. 
Estas atualizações são depois solicitas por uma página \emph{Web} desenvolvida para a visualização da informação que se encontra no nó de visualização, as quais são usadas para
o desenho de um grafo que representa o diretório e tabelas que dispõem a ordem das tabelas e dos elementos das filas. 
Para além de dispor o estado do sistema, esta página é também interativa, na qual é possível forçar \emph{Nodes} do sistema a realizar pedidos de acesso, ou controlar a animação.

Por fim, ordem de reconstrução dos eventos foi necessária para provar (parcialmente) o funcionamento da implementação feita.
Neste sistema existem dois tipos de acontecimento no sistema que têm de seguir a mesma ordem, estes sendo,
a entrada de um nó na fila ``Principal'', fila de espera que está ligada ao nó detentor do objeto, e a chegada do objeto a um nó e foi necessário
reconstruir esses mesmos acontecimentos usando apenas as atualizações, estas que não seguiam a ordem de acontecimento nem faziam uso de \emph{timestamps} para as sincronizar. 
Então foi necessário usar outros fatores no qual há a certeza da ordem, como a mudança do tipo de um nó, ou o facto de se conhecer que um nó está à espera de outro para obter o acesso
ao objeto para ser possível a reconstrução da ordem pela qual os nós entram na fila e a ordem que o (acesso ao) objeto segue.


\section{Trabalho Futuro}
\label{sec:trab-futuro}
% TODO: mudar "parte"
A parte gráfica desenvolvida neste trabalho poderia ser utilizada na visualização de outros algoritmos como,
por exemplo, os trabalhos referidos na secção \ref{motivacao:sec:trabalhos_relacionados}
, o ``Ivy'' e ``Arvy'', pois esta foi implementada de mode a ser independente do algoritmo ``Arrow'',
e assim também seria interessante implementar esses mesmos algoritmos e fazer comparações entre estes. \\

\noindent O programa do \emph{Node} poderia ser alterado de modo a ser utilizado como uma biblioteca externa,
e aplicado em Sistemas de Ficheiros Distribuídos ou Bases de Dados Distribuídas,
em que as várias Bases de Dados partilhariam documentos/informações fazendo seguindo o protocolo estudado,
permitindo o acesso exclusivo a um documento/informação e evitando que algumas das Bases de Dados se tornassem em ``Hotspots''. \\

\noindent Também seria interessante a adaptar o trabalho atual para, ao invés de ser partilhado apenas um objeto, fosse possível que existissem vários diretórios, 
e que para tal seria necessário que as ligações entre os nós fossem formadas através de um algoritmo,
pois seria trabalhoso especificar manualmente as ligações para cada diretório. \\
% era possível pegar na parte gráfica e implementar outros algoritmos do género
% aplicar o algoritmo num contexto real, como uma database ou um filesystem distribuídos,


