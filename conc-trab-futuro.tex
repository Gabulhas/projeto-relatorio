\chapter{Conclusões e Trabalho Futuro}
\label{chap:conc-trab-futuro}

\section{Conclusões Principais}
\label{sec:conc-princ}
% falar que o projeto é diferente do enunciado
% falar que foi possível fazer um trabalho distribuído ao invés de apenas concorrente
% foi possível fazer uma implementação que não exisitia e distribuído

% Esta secção contém a resposta à questão: \\
% \emph{Quais foram as conclusões princípais a que o(a) aluno(a) chegou no fim deste trabalho?}
Neste trabalho foi possível estudar e implementar o protocolo/algoritmo ``Arrow Distributed Directory Protocol'', sendo que no desenvolvimento deste foi decidido implementar-se num paradigma diferente ao descrito no enunciado deste projeto, isto é, o sistema funciona num contexto distribuído, e não apenas concorrente e numa só máquina. 
A implementação da componente gráfica teve a utilidade na compreensão do algoritmo, e facilitou a visualização e teste da implementação do programa \emph{Node}. Com esta é possível observar num único sítio/numa única interface e reconstruir o estado de um sistema distribuído desenvolvido através de apenas das atualização de cada nó do sistema.


%\section{Reflexão Crítica}
% falar de como o trabalho foi difícil de implementar por ser distribuído/concorrente 
% falar de como a implementação podia ter sido doutra forma, principalmente a visualização


%TODO: remover?
\section{Escolhas de Implementação}
\label{sec:escolhas-implementacao}
Como referido na secção \ref{sec:conc-princ}, escolheu-se fazer uma implementação distribuída, ao contrário do decidido inicialmente, porque este seria o contexto onde este algoritmo seria aplicado e também tornaria a realização deste projeto mais interessante e desafiante.
Quanto à visualização, optou-se por se usar uma página \emph{Web}, pois é a forma mais indicada de renderizar o sistema caso o \emph{Node} de visualização estivesse a ser executado numa máquina remota e possivelmente sem monitor, que é o esperado para este sistema.
% explicar o porquê de se ter implementado distribuído ao invés de cocncorrente
% explicar o uso de uma interface gráfica **e** numa págian web


\section{Trabalho Futuro}
\label{sec:trab-futuro}
% TODO: mudar "parte"
A parte gráfica desenvolvida neste trabalho poderia ser utilizada na visualização de outros algoritmos como, por exemplo, os trabalhos referidos na secção \ref{motivacao:sec:trabalhos_relacionados}
, o ``Ivy'' e ``Arvy'', pois esta foi implementada de uma forma independente do algoritmo ``Arrow'', e assim também seria interessante implementar esses mesmos algoritmos e fazer comparações entre estes.
O programa do \emph{Node} poderia ser alterado de forma a ser utilizado como uma biblioteca externa, e aplicado em Sistemas de Ficheiros Distribuídos ou Bases de Dados Distribuídas, em que as várias Bases de Dados partilhariam documentos/informações fazendo seguindo o protocolo estudado, permitindo o acesso exclusivo a um documento/informação e evitando que algumas das Bases de Dados se tornassem em ``Hotpots''.
% era possível pegar na parte gráfica e implementar outros algoritmos do género
% aplicar o algoritmo num contexto real, como uma database ou um filesystem distribuídos,


% Esta secção responde a questões como:\\
% \emph{O que é que ficou por fazer, e porque?}\\
% \emph{O que é que seria interessante fazer, mas não foi feito por não ser exatamente o objetivo deste trabalho?}\\
%  \emph{Em que outros casos ou situações ou cenários -- que não foram estudados no contexto deste projeto por não ser seu objetivo -- é que o trabalho aqui descrito pode ter aplicações interessantes e porque?}
