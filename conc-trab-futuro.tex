\chapter{Conclusões e Trabalho Futuro}
\label{chap:conc-trab-futuro}

\section{Conclusões Principais}
\label{sec:conc-princ}

\begin{comment}
    Lições aprendidas
    - Implementaçã de um sistema concorrente e distribuído
    - Evitar race conditions
    - Oportunidade de usar Mecanismos de Sincronização e Concorrência

    Pontos Principais
    - o que é que foi possível fazer
    - o que é em geral este trabalho/implementação
    - possível implementação moderna com visualização do algoritmo 

    Neste trabalho foi possível estudar e implementar o sistema concorrente e distribuído, que faz uso do ``Arrow Distributed Directory Protocol''.
    O que torna notável este protocolo é o mecanismo de inversão ou troca de ligações, que mesmo sendo simples,
    evita que mais que um elemento no sistema tenha acesso a um objeto que pretendemos ter o acesso exclusivo, isto é, evita falhas de Condições de Corrida.
    Este tipo de falhas pode ter grande impacto no bom funcionamento de programas concorrentes, pois a leitura e/ou escrita concorrente de 
    informação pode levar a resultados imprevisíveis. 

    Após um estudo do o artigo que introduz o \acs{ADDP} \cite{Arrow}, que permitiu a realização de uma especificação dos elementos que pertencem ao sistema definido por o protocolo estudado,
    esta que foi usada como base no planeamento e implementação do projeto.
    No desenvolvimento do programa executado no sistema distribuído, foi possível praticar o uso de mecanismos de sincronização e concorrentes, estes oferecidos pela Linguagem \emph{Go},
    a linguagem escolhida para a implementação.
    Para além da implementação do programa distribuído, foi desenvolvida uma interface, que permitiu visualizar o protocolo em ação, testar a implementação, e provar (parcialmente) o seu
    bom funcionamento e fidelidade com o descrito na Especificação.
\end{comment}


Este projeto teve como objetivo principal o estudo e implementação do \acs{ADDP}, um protocolo que permite a partilha de um objeto móvel, sem ser necessário
que todos os nós do sistema detenham da infomação da localizam deste e que garante o acesso mutuamente exclusivo a este. \\


Foi possível a exploração de vários tópicos como concorrência, protocolos de diretórios, objetos móveis, sincronização de processos concorrentes,
sistemas e estruturas de dados distribuídas, e exclusão mútua distribuída. \\

Após um estudo do o artigo que introduz o \acs{ADDP} \cite{Arrow}, que permitiu a realização de uma especificação dos elementos que pertencem ao sistema definido por o protocolo estudado,
esta que foi usada como base no planeamento e implementação do projeto.
No desenvolvimento do programa executado no sistema distribuído, foi possível praticar o uso de mecanismos de sincronização e concorrentes, estes oferecidos pela Linguagem \emph{Go},
a linguagem escolhida para a implementação.
Para além da implementação do programa distribuído, foi desenvolvida uma interface, que permitiu visualizar o protocolo em ação, testar a implementação, e provar (parcialmente) o seu
bom funcionamento e fidelidade com o descrito na Especificação.


% Esta secção contém a resposta à questão: \\
% \emph{Quais foram as conclusões principais a que o(a) aluno(a) chegou no fim deste trabalho?}


\section{Trabalho Futuro}
\label{sec:trab-futuro}
% TODO: mudar "parte"
A parte gráfica desenvolvida neste trabalho poderia ser utilizada na visualização de outros algoritmos como,
por exemplo, os trabalhos referidos na secção \ref{motivacao:sec:trabalhos_relacionados}
, o ``Ivy'' e ``Arvy'', pois esta foi implementada de mode a ser independente do algoritmo ``Arrow'',
e assim também seria interessante implementar esses mesmos algoritmos e fazer comparações entre estes. \\

\noindent O programa do \emph{Node} poderia ser alterado de modo a ser utilizado como uma biblioteca externa,
e aplicado em Sistemas de Ficheiros Distribuídos ou Bases de Dados Distribuídas,
em que as várias Bases de Dados partilhariam documentos/informações fazendo seguindo o protocolo estudado,
permitindo o acesso exclusivo a um documento/informação e evitando que algumas das Bases de Dados se tornassem em ``Hotspots''. \\

\noindent Também seria interessante a adaptar o trabalho atual para, ao invés de ser partilhado apenas um objeto, fosse possível que existissem vários diretórios, 
e que para tal seria necessário que as ligações entre os nós fossem formadas através de um algoritmo,
pois seria trabalhoso especificar manualmente as ligações para cada diretório. \\
% era possível pegar na parte gráfica e implementar outros algoritmos do género
% aplicar o algoritmo num contexto real, como uma database ou um filesystem distribuídos,


