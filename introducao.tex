\chapter{Introdução}
\label{chap:intro}
% explicar que o mundo cada vez mais requer programas distribuídos devido à escalabilidade destes
% aumento de cores no CPU
% aumento da infraestruturas das grandes empresas
% durabilidade dos sistemas
% maior runtime
% contra failures
% A internet é cada vez mais global, logo é necessário distribuir os sistemas por várias partes do mundo
\section{Enquadramento}
\label{sec:amb} 
% foco do trabalho -> Estruturas de dados (queue, MST, Nodes), Concurrência, Rede distribuida (principal),


% mudar palavras

Este projeto envolve o estudo e implementação de um protocolo para sistemas distribuídos, o \textit{Arrow Distributed Directory Protocol} (ADDP) \cite{Arrow}. 
A realização deste compreende os tópicos de programação em Linguagem \emph{Go}, Estruturas de Dados, concorrência, sistemas e algoritmos distribuídos.

A elaboração deste projeto decorreu-se na unidade de projeto de final de curso da Licenciatura em Engenharia Informática.



\section{Objetivos}
\label{sec:obj}
Inicialmente foram estabelecidos objetivos, no entanto durante a implementação do projeto, o tema distanciou-se do inicialmente proposto. 
Os objetivos no desenvolvimento do projeto foram:

\begin{itemize}
    \item Leitura do \emph{paper} oirignal e proposto e compreensão geral do protocolo.
    \item Elaboração da espeficiação dos elementos, ligações entre estes e os seus comportamentos.
    \item Implementação inicial do programa.
    \item Execução de várias instâncias do programa com o uso da ferramenta \emph{Docker}.
    \item Implementação da visualização do grafo representativo do diretório.
    \item Implementação da visualização da fila (\emph{Queue}).

\end{itemize}


\section{Organização do Documento}
\label{sec:organ}
% fazer ao final
\begin{enumerate}
\item 
\item 
\item 
\end{enumerate}
