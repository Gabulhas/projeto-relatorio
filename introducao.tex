\chapter{Introdução}
\label{chap:intro}
% explicar que o mundo cada vez mais requer programas distribuídos devido à escalabilidade destes
% aumento de cores no CPU
% aumento da infraestruturas das grandes empresas
% durabilidade dos sistemas
% maior runtime
% contra failures
% A internet é cada vez mais global, logo é necessário distribuir os sistemas por várias partes do mundo
\section{Enquadramento}
\label{sec:amb} 
% foco do trabalho -> Estruturas de dados (queue, MST, Nodes), Concurrência, Rede distribuida (principal),


% mudar palavras

Este projeto envolve o estudo e implementação de um protocolo para sistemas distribuídos, o \textit{Arrow Distributed Directory Protocol} (ADDP) \cite{Arrow}. 
A realização deste compreende os tópicos de programação em Linguagem \emph{Go}, Estruturas de Dados, concorrência, sistemas e algoritmos distribuídos.

A elaboração deste projeto decorreu-se na unidade de projeto de final de curso da Licenciatura em Engenharia Informática.



\section{Objetivos}
\label{sec:obj}
Inicialmente foram estabelecidos objetivos, no entanto durante a implementação do projeto, o tema distanciou-se do inicialmente proposto. 
Os objetivos no desenvolvimento do projeto foram:

\begin{itemize}
    \item Leitura do \emph{paper} original e proposto e compreensão geral do protocolo.
    \item Elaboração da especificação dos elementos, ligações entre estes e os seus comportamentos.
    \item Implementação inicial do programa.
    \item Execução de várias instâncias do programa com o uso da ferramenta \emph{Docker}.
    \item Implementação da visualização do grafo representativo do diretório.
    \item Implementação da visualização da fila (\emph{Queue}).
    \item Implementação da visualização do histórico do elementos da fila e detentores do acesso ao objeto.

\end{itemize}

\section{Resultados Atingidos}
Serão descritos os resultados atingidos no desenvolvimento do projeto.

\begin{itemize}
    \item Foi elaborada uma especificação de todos os elementos pertencentes ao sistema, tendo como maior foco os comportamentos dos nós no sistema. Esta especificação é também descrita neste relatório. 
    \item Desenvolveu-se o programa dos nós, sendo o objetivo de maior importância na elaboração deste projeto, pois um conjunto destes programas representam um sistema que segue o protocolo \emph{Arrow}. Para além disso, incluído nesta implementação temos:

    \begin{itemize}
	\item Implementação dos comportamentos dos nós.
	\item Comunicação entre nós através do protocolo \acs{HTTP}.
	\item As transformações que os nós podem sofrer.
	\item Definição das mensagens transmitidas entre os nós.
    \end{itemize}

\item Implementação de uma visualização gráfica das estruturas de dados distribuídas presentes na rede, como um grafo que representa o diretório da rede e uma tabela que representa a fila de chegada do acesso.
\item O uso da ferramenta \emph{Docker} para a execução de várias instâncias do programa dos nós de forma automática e distribuída.
\item A implementação de uma ferramenta de auxílio para a transformação de uma topologia/definição de uma rede no formato \acs{CSV} para o formato ``docker-compose.yml'', usado como ficheiro de execução da ferramenta \emph{Docker}.

\end{itemize}




\section{Organização do Documento}
\label{sec:organ}
%alterar
Este documento está organizado da seguinte forma:
\begin{enumerate}
    \item \textbf{Introdução} - Citação do artigo de referência usado para a implementação, os objetivos do projeto e os resultados atingidos no mesmo.
    \item \textbf{Motivação} - Apresentação das vantages do uso de sistema distribuídos, tais como complexidade do desenvolvimento dos mesmos. Descrição do protocolo de referência e citação de trabalhos relacionados.
    \item \textbf{Especificação} 
\item 
\end{enumerate}
